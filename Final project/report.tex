\documentclass[11pt,a4paper]{article}

\usepackage{latexsym}
\usepackage{graphicx}
\usepackage[french]{babel}


\usepackage{amsmath,amssymb}
\usepackage{pstricks,pst-plot}
\usepackage{calc}
\usepackage{multicol}
\usepackage{fancyhdr}
\usepackage{lastpage}
\usepackage[T1]{fontenc}
\usepackage[utf8]{inputenc}  
\usepackage{lmodern}
\usepackage{stmaryrd}
\usepackage[]{algorithm2e}
\usepackage{float}

\pagestyle{plain}

\title{ADVANCED LEARNING FOR TEXT AND GRAPH DATA \\ Final Project: Text categorization}
\author{Mathurin \textsc{Massias} \and Clément \textsc{Nicolle} \and Michaël Weiss}
\date{\today} 


\begin{document}
	
\maketitle

\section{Bag-of-words model}

\subsection{Description and classification algorithms}

We first implemented the bag-of-words model. From the train and test datasets, two matrices of size $n\_documents \times n\_terms$ were computed, using the dictionary of terms present in the train dataset. Then, each document is represented by a vector in dimension $n\_terms$ (a line of the matrix).
\\The training set contains 14,575 different words in 5485 documents. THe test set is composed of 2,189 documents.
\\We tried out several classification algorithms on the feature dataset:
\begin{itemize}
	\item k-Nearest Neighbors: for a given document of the test dataset, we find out the k closest documents in the train dataset using cosine similarity. Then, we assign to the test document the most frequent label among these k neighbors. Arbitrarily, in case of tie, the first label in alphabetic order among the most frequent will be picked up.
	\item Support Vector Machine: we used a Gaussian kernel, and cross-validation was made in order to select best values for parameters C and gamma.
	\item Random Forest: we made a cross-validation to select the optimal number of trees to compute.
	\item Adaboost: we also made a cross-validation for the number of boundaries here.
\end{itemize}


\subsection{Results}

As suggested in the subject, we used micro-averaging and macro-averaging precision and recall in order to evaluate the performances of our classifiers. We can notice that micro-averaging precision and recall are indeed the same number.
\\Cross-validations were made upstream in order to find the optimal values for the parameters: k for k-NN, C and $\gamma$ for RBF-SVM, number of trees for random forests. With these parameters fitted, here are the different performances the algorithms achieved :

\begin{table}[h]
\hspace*{-17mm}	\begin{tabular}{|l|c|c|c|c|}
		\hline
		\multicolumn{1}{|c|}{Algorithm \textbackslash Performance} & \begin{tabular}[c]{@{}c@{}}Micro-averaging\\ precision/recall\end{tabular} & \begin{tabular}[c]{@{}c@{}}Macro-averaging\\ precision\end{tabular} & \begin{tabular}[c]{@{}c@{}}Macro-averaging\\ recall\end{tabular} & Training time \\ \hline
		k-NN                                        & 84.42\%                                                                    & 85.40\%                                                             & 82.74\%                                                          & 1543 s           \\ \hline
		SVM                                         & 89.31\%                                                                    & 92.63\%                                                             & 65.06\%                                                          & 1021 s           \\ \hline
		Random Forest                               & 91.6\%                                                                     & 89.46\%                                                             & 62.84\%                                                          & 7 s           \\ \hline
		Adaboost                                    & 79.1\%                                                                    & 57.02\%                                                             & 57.86\%                                                          & 204 s           \\ \hline
	\end{tabular}
	\caption{Time and classification performance for various algorithms on the bag-of-words model}
	\end{table}
For algorithms with a part of randomness (RF), the results are averaged over 10 repetitions.
\\[5mm]The second quickest, Adaboost performs poorly compared to other algorithms. 
All other algorithms have very good classification performance. SVM takes time to train, because we use a "one-vs-one" multiclass SVM (needing 14*13/2 binary SVM training). k-NN does not require training, but each classification requires computation of 5,485 cosine distances in dimension 14,575, which takes approximately one second.
\\Though slowly inferior in terms of macro averaging recall, Random Forest is by far the fastest algorithm (150 times quicker!).
\\Depending on the trade-off between training time and recall, we would recommend using either a one-vs-one multiclass SVM or a Random Forest for the bag-of-words model.

\section{Graph-of-words model}
The we implemented a Graph-of-words model. As the subject suggested it we firstly tried to implement it with windows of size $4$ to find the edges. This allowed us to have better classification results overall. \\
The two measure we associated to the graph nodes were degree centrality measure and eigenvector centrality measure.
The two models of graphs we used were undirected unweigthed graphs, and undirected unweighted multigraphs of the library networkx. The second one allows us to have multiple same edge between $2$ nodes and therefore to have an equivalent of an undirected weighted graph.\\
Since after the graphs were constructed, the final input to use for the classification algorithms remained a document-term matrix of the same size than the one obtained in the bag-of-words model, we applied the same classification algorithms in order to compare the results.\\
The same way the score of words per document were penalized by the factor $IDF$ in the bag-of-words model, we also penalized the document-term matrix by the $IDF$ factors.
\subsection{Graphs with unweighted edges}
\paragraph{Degree Centrality measure \newline}
The time of execution needed to compute the document term matrix (and the graphs for every document before) was approximately of 300s. This time was fast enough to allows us to test several different parameters to try to find the best approach for our classification.

\begin{table}[H]
	\begin{tabular}{|l|c|c|c|}
		\hline
		\multicolumn{1}{|c|}{Algorithm \textbackslash Performance} & \begin{tabular}[c]{@{}c@{}}Micro-averaging\\ precision/recall\end{tabular} & \begin{tabular}[c]{@{}c@{}}Macro-averaging\\ precision\end{tabular} & \begin{tabular}[c]{@{}c@{}}Macro-averaging\\ recall\end{tabular} \\ \hline
		k-NN                                        & 86.16\%                                                                    & 83.64\%                                                             & 82.22\%           \\ \hline
		SVM                                         & 95.16\%                                                                    & 95.44\%                                                             & 79.89\%            \\ \hline
		Random Forest                               & 91.09\%                                                                     & 86.05\%                                                             & 66.07\%             \\ \hline
		Adaboost                                    & 79.53\%                                                                    & 67.18\%                                                             & 67.36\%              \\ \hline
	\end{tabular}
	\caption{Classification performance by algorithm for Degree Centrality measure}
\end{table}

\paragraph{Eigenvector Centrality measure \newline}
Here, using the networkx eigenvector centrality measure the time of execution needed to compute the document term matrix was around $450$s.  
\begin{table}[H]
	\begin{tabular}{|l|c|c|c|}
		\hline
		\multicolumn{1}{|c|}{Algorithm \textbackslash Performance} & \begin{tabular}[c]{@{}c@{}}Micro-averaging\\ precision/recall\end{tabular} & \begin{tabular}[c]{@{}c@{}}Macro-averaging\\ precision\end{tabular} & \begin{tabular}[c]{@{}c@{}}Macro-averaging\\ recall\end{tabular} \\ \hline
		k-NN                                        & 86.71\%                                                                    & 82.53\%                                                             & 81.91\%           \\ \hline
		SVM                                         & 93.24\%                                                                    & 94.11\%                                                             & 69.66\%            \\ \hline
		Random Forest                               & 91.65\%                                                                     & 88.95\%                                                             & 68.47\%             \\ \hline
		Adaboost                                    & 79.03\%                                                                    & 66.49\%                                                             & 60.98\%              \\ \hline
	\end{tabular}	
	\caption{Classification performance by algorithm for Eigenvector Centrality measure}

\end{table}

Therefore using the Eigenvector centrality measure does not give any remarkable improvement. The precisions score overall remain the same.
\subsection{Graph with Weighted Edges}
With the multigraph structure, we had a particularly small time of execution needed to compute the document term matrix which was of $120$s. Of course, we added every edge to the multigraph without checking if the edge already existed as it is done in the undirect graph structure we used previously. Since the eigenvector centrality measure did not seem to greatly improve our classification measure, we only tried this structure associated to a degree centrality measure.
\begin{table}[H]
	\begin{tabular}{|l|c|c|c|}
		\hline
		\multicolumn{1}{|c|}{Algorithm \textbackslash Performance} & \begin{tabular}[c]{@{}c@{}}Micro-averaging\\ precision/recall\end{tabular} & \begin{tabular}[c]{@{}c@{}}Macro-averaging\\ precision\end{tabular} & \begin{tabular}[c]{@{}c@{}}Macro-averaging\\ recall\end{tabular} \\ \hline
		k-NN                                        & 86.25\%                                                                    & 86.96\%                                                             & 81.55\%           \\ \hline
		SVM                                         & 95.25\%                                                                    & 95.38\%                                                             & 79.62\%            \\ \hline
		Random Forest                               & 91.82\%                                                                     & 90.58\%                                                             & 69.33\%             \\ \hline
		Adaboost                                    & 79.21\%                                                                    & 69.01\%                                                             & 65.97\%              \\ \hline
	\end{tabular}
	\caption{Classification performance by alogrithm for Weighted Edges with Degree Centrality measure}
\end{table}


\subsection{Influence of the size of the sliding window} 


\end{document}